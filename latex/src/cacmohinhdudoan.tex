
\section{Các mô hình dự đoán}

Nhóm sử dụng bốn thuật toán học máy phổ biến để giải quyết bài toán dự đoán nguy cơ mắc Alzheimer:
\begin{itemize}
    \item \textbf{Decision Tree (Cây quyết định)}
    \item \textbf{K-Nearest Neighbors (KNN)}
    \item \textbf{Logistic Regression (Hồi quy Logistic)}
    \item \textbf{Support Vector Machine (SVM)}
\end{itemize}

\subsection*{Mô tả các mô hình}
\begin{itemize}
    \item \textbf{Decision Tree:} Mô hình phân loại dựa trên cấu trúc cây, dễ giải thích, cho phép xác định các đặc trưng quan trọng. Tham số chính: \texttt{max\_depth} (độ sâu tối đa của cây).
    \item \textbf{KNN:} Phân loại dựa trên số lượng láng giềng gần nhất. Tham số chính: \texttt{n\_neighbors} (số láng giềng).
    \item \textbf{Logistic Regression:} Mô hình tuyến tính cho bài toán phân loại nhị phân. Tham số chính: \texttt{C} (hệ số điều chỉnh regularization).
    \item \textbf{SVM:} Tìm siêu phẳng phân tách tối ưu giữa hai lớp. Tham số chính: \texttt{C} (điều chỉnh regularization), \texttt{gamma} (tham số kernel).
\end{itemize}

\subsection*{Quy trình huấn luyện và đánh giá}
\begin{enumerate}
    \item Chia dữ liệu thành tập huấn luyện (80\%) và kiểm thử (20\%).
    \item Sử dụng \textbf{GridSearchCV} để tìm bộ tham số tối ưu cho từng mô hình với 5-fold cross-validation.
    \item Đánh giá mô hình trên tập kiểm thử bằng các chỉ số: accuracy, precision, recall, F1-score.
    \item So sánh kết quả giữa các mô hình để chọn ra phương pháp phù hợp nhất.
\end{enumerate}


Kết quả chi tiết và so sánh hiệu quả các mô hình sẽ được trình bày ở phần tiếp theo.

	extbf{Ví dụ code huấn luyện và đánh giá:}
\begin{lstlisting}[language=Python]
from sklearn.model_selection import GridSearchCV
from sklearn.metrics import classification_report

param_grid = {'max_depth': [3, 5, 7, 12, None]}
grid = GridSearchCV(DecisionTreeClassifier(), param_grid, cv=5, scoring='accuracy')
grid.fit(X_train, y_train)
y_pred = grid.best_estimator_.predict(X_test)
print(classification_report(y_test, y_pred))
\end{lstlisting}
