\newpage

\section{Giới thiệu bài toán và bộ dữ liệu}

Bệnh Alzheimer là một trong những nguyên nhân hàng đầu gây sa sút trí tuệ ở người cao tuổi, ảnh hưởng nghiêm trọng đến chất lượng cuộc sống và tạo gánh nặng lớn cho xã hội. Việc phát hiện sớm nguy cơ mắc Alzheimer giúp nâng cao hiệu quả điều trị, giảm chi phí và kéo dài thời gian sống độc lập cho bệnh nhân. Do đó, xây dựng các mô hình học máy dự đoán nguy cơ mắc Alzheimer dựa trên dữ liệu lâm sàng là một hướng tiếp cận thực tiễn và cấp thiết.


\subsection{Giới thiệu bộ dữ liệu}
Bộ dữ liệu sử dụng là \textbf{alzheimers\_disease\_data.csv} với 2.149 bệnh nhân và 35 thuộc tính, bao gồm: nhân khẩu học (tuổi, giới tính, dân tộc, trình độ học vấn), chỉ số sinh học (BMI, huyết áp, cholesterol...), tiền sử bệnh lý (tim mạch, tiểu đường, tăng huyết áp...), các triệu chứng lâm sàng (suy giảm trí nhớ, rối loạn hành vi, khả năng thực hiện hoạt động thường ngày, v.v.) và nhãn chẩn đoán Alzheimer (Diagnosis).

	extbf{Lý do chọn bộ dữ liệu:}
\begin{itemize}
    \item Dữ liệu đa dạng, đầy đủ các yếu tố nguy cơ và triệu chứng liên quan đến Alzheimer.
    \item Không có giá trị thiếu, thuận lợi cho việc tiền xử lý và xây dựng mô hình.
    \item Phù hợp với mục tiêu xây dựng mô hình dự đoán nhị phân (có/không mắc bệnh).
\end{itemize}

	extbf{Input:} Hồ sơ bệnh nhân với các đặc trưng như trên.

	extbf{Output:} Nhãn dự đoán nguy cơ mắc Alzheimer (0: Không mắc, 1: Mắc bệnh).


\subsection{Lựa chọn các mô hình học máy}
Nhóm lựa chọn 4 mô hình học máy phổ biến, đại diện cho các hướng tiếp cận khác nhau:
\begin{itemize}
    \item \textbf{Decision Tree} (Cây quyết định): Mô hình trực quan, dễ giải thích, phù hợp với dữ liệu có nhiều biến phân loại.
    \item \textbf{K-Nearest Neighbors (KNN)}: Đơn giản, hiệu quả với dữ liệu không quá lớn, không giả định phân phối dữ liệu.
    \item \textbf{Logistic Regression} (Hồi quy Logistic): Mô hình tuyến tính, dễ triển khai, cho phép đánh giá mức độ ảnh hưởng của từng đặc trưng.
    \item \textbf{Support Vector Machine (SVM)}: Mô hình mạnh mẽ, hiệu quả với dữ liệu có biên phân tách rõ ràng, hỗ trợ kernel cho các bài toán phi tuyến.
\end{itemize}
Việc lựa chọn này giúp so sánh hiệu quả giữa các thuật toán truyền thống, từ đó chọn ra phương pháp phù hợp nhất cho bài toán dự đoán nguy cơ mắc Alzheimer.


\section{Tổng quan phương pháp}
Quy trình thực hiện gồm các bước:
\begin{itemize}
    \item Phân tích, khám phá dữ liệu (EDA) để hiểu rõ đặc trưng và mối quan hệ giữa các biến.
    \item Tiền xử lý dữ liệu: làm sạch, chuẩn hóa, mã hóa biến phân loại, xử lý ngoại lai.
    \item Xây dựng, huấn luyện và tối ưu các mô hình học máy.
    \item Đánh giá, so sánh hiệu quả các mô hình bằng các chỉ số: accuracy, recall, specificity, F1-score.
\end{itemize}
Chi tiết từng bước sẽ được trình bày ở các phần tiếp theo.