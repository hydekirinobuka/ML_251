\newpage

\section*{Tiền xử lí dữ liệu}

\subsection*{Biến liên tục}
Các biến liên tục như tuổi, BMI, các chỉ số huyết áp, cholesterol, điểm MMSE, ADL,... được chuẩn hóa để đưa về cùng thang đo, giúp mô hình học hiệu quả hơn. Hai phương pháp phổ biến được sử dụng:
\begin{itemize}
    \item \textbf{Normalization (Min-Max Scaling):} Đưa giá trị về khoảng [0, 1].
    \item \textbf{Standardization (Z-score):} Đưa dữ liệu về phân phối chuẩn với trung bình 0, độ lệch chuẩn 1.
\end{itemize}

\textbf{Xử lý giá trị ngoại lai (outliers):}
\begin{itemize}
    \item Sử dụng phương pháp IQR (Interquartile Range) hoặc z-score để phát hiện và loại bỏ/điều chỉnh các giá trị ngoại lai ở các biến số như BMI, huyết áp, cholesterol.
    \item Trong thực tế, bộ dữ liệu này đã được làm sạch nên số lượng ngoại lai không đáng kể.
\end{itemize}

\subsection*{Biến rời rạc (không có ý nghĩa thứ tự)}
Các biến phân loại (categorical) như giới tính, dân tộc, trình độ học vấn, hút thuốc, tiền sử bệnh lý,... được mã hóa để mô hình có thể xử lý:
\begin{itemize}
    \item \textbf{One-Hot Encoding:} Biến phân loại nhiều giá trị (ví dụ: Ethnicity) được chuyển thành các cột nhị phân.
    \item \textbf{Label Encoding:} Biến nhị phân (Yes/No, Nam/Nữ) được mã hóa thành 0/1.
\end{itemize}

\subsection*{Chia dữ liệu huấn luyện và kiểm thử}
\begin{itemize}
    \item Dữ liệu được chia thành hai tập: 80\% dùng để huấn luyện (train), 20\% dùng để kiểm thử (test).
    \item Sử dụng hàm \texttt{train\_test\_split} của thư viện scikit-learn với tham số \texttt{random\_state=42} để đảm bảo kết quả có thể tái lập.
\end{itemize}

	extbf{Ví dụ code:}
\begin{lstlisting}[language=Python]
from sklearn.model_selection import train_test_split
X_train, X_test, y_train, y_test = train_test_split(X, y, test_size=0.2, random_state=42)
\end{lstlisting}

Sau khi tiền xử lý, dữ liệu đã sẵn sàng cho quá trình huấn luyện và đánh giá các mô hình học máy.